%!TEX root = ../../../main.tex

\toggletrue{image}
\toggletrue{imagehover}
\chapterimage{query_2}
\chapterimagetitle{\uppercase{Query}}
\chapterimageurl{https://xkcd.com/1409/}
\chapterimagehover{Teil 2 des Comics.}

\chapter{\acs{SQL}-Abfragen: Aggregation und Gruppierung}
\label{chapter-sql-abfragen-aggregation-gruppierung}

In diesem Kapitel erweitern wir die bereits kennengelernte Datenbankabfragen. Wir schauen uns Aggregatfunktionen\footnote{In der Mathematik versteht man unter Aggregation die Zusammenfassung von Daten nach mathematisch-statistischen Methoden.} an, mit denen wir Werte zusammenfassen können. Wir können beispielsweise den Durchschnitt ausrechnen lassen. Als zweite Erweiterung betrachten wir das Gruppieren von Zeilen. Das Lernziel lautet:

\newcommand{\sqlAbfragenAggregationGruppierungLernziele}{
\protect\begin{todolist}
\item Sie erstellen \acs{SQL}-Abfragen mit Aggregatfunktionen.
\item Sie erstellen \acs{SQL}-Abfragen mit Gruppierungen.
\item Sie erstellen \acs{SQL}-Abfragen mit einem Filter für Gruppen.
\end{todolist}
}

\lernziel{\autoref{chapter-sql-abfragen-aggregation-gruppierung}, \nameref{chapter-sql-abfragen-aggregation-gruppierung}}{\protect\sqlAbfragenAggregationGruppierungLernziele}

\sqlAbfragenAggregationGruppierungLernziele

\section{Aggregation}

Wir können Daten zusammenführen, in dem wir Aggregatfunktionen verwenden. Es stehen folgende Funktionen zur Verfügung:

\begin{itemize}
\item \lstinline[language=SQL, upquote=true]{count(A)}: Zählt die Anzahl aller Werte in der Spalte $A$.
\item \lstinline[language=SQL, upquote=true]{avg(A)}: Berechnet den Durchschnitt aller Werte in der Spalte $A$.
\item \lstinline[language=SQL, upquote=true]{sum(A)}: Berechnet die Summe aller Werte in der Spalte $A$.
\item \lstinline[language=SQL, upquote=true]{min(A)}/\lstinline[language=SQL, upquote=true]{max(A)}: Berechnet das Minimum/Maximum aller Werte in der Spalte $A$.
\end{itemize}

\begin{example}[Das Durchschnittsrating aller Seeleute bestimmen]
Das Ergebnis dieser Abfrage ist eine Zeile mit einer Spalte und dem Wert $6.6364$.
\begin{lstlisting}[language=SQL, upquote=true]
select avg(rating)
from sailor
\end{lstlisting}
\end{example}

Wenn wir \lstinline{distinct} verwenden, dann werden vor der Ausführung alle \textbf{Duplikate} eliminiert. 

\begin{example}
Jedes Rating wird nur einmal bei der Berechnung verwendet.
\begin{lstlisting}[language=SQL, upquote=true]
select avg(distinct rating)
from sailor
\end{lstlisting}
\end{example}

\subsection{Spalten umbenennen}

Wir können im Ergebnis eine Spalte auch umbenennen. Dies geschieht \say{on-the-fly}: Die Tabelle wird nicht verändert, es findet nur eine \textbf{Umbenennung im Ergebnis} statt.

\begin{example}[Die Anzahl aller Seeleute auflisten]
Wir benennen die Spalte zu Anzahl um.
\begin{lstlisting}[showstringspaces=off, language=SQL]
select count(sailor_id) as anzahl
from sailor
\end{lstlisting}
\end{example}

Wir können zusätzlich auch noch die \lstinline[language=SQL]{where}-Komponente verwenden. Die Zeilen werden \textbf{vor} der Aggregatfunktion eliminiert. Das Umbenennen einer Spalte kann auch ohne Aggregatfunktion erfolgen.

\subsection{Stolperfallen}

Es ist im Allgemeinen \textbf{nicht} sinnvoll, neben der \textbf{Aggregatfunktion} noch weitere Spalten im \lstinline[language=SQL]{select} anzeigen zu lassen. MariaDB zeigt meist einfach den Wert aus der ersten Zeile an.

\begin{example}[Nicht sinnvoll]
Da eine Zahl berechnet wird, ist es nicht sinnvoll auch noch den Namen anzuzeigen. Es ist nicht klar, welcher Name angezeigt werden soll. Entweder verweigert das \ac{DBMS} die Ausführung oder es wird ein beliebiger Name angezeigt.
\begin{lstlisting}[showstringspaces=off, language=SQL]
select sailor_name, count(sailor_id) as anzahl
from sailor
\end{lstlisting}
\end{example}

\begin{important}
Der spezielle Wert \lstinline{null} wird bei (den meisten) Aggregatfunktionen \textbf{vor} der Ausführung der Funktion eliminiert.
\end{important}

Dieses Verhalten ist sinnvoll. Das Summieren ist beispielsweise auch nicht sinnvoll mit \lstinline{null}. Was soll \lstinline{null} plus 1 ergeben?

\section{Aufgaben}

\subsection{Aufgabe 1}

Fügen Sie die Tabelle \lstinline{mitarbeiter} zur Ihrer Datenbank hinzu. Sie finden die passenden \ac{SQL}-Befehle in einer Textdatei in Microsoft Teams. Sie sehen hier die Struktur der Tabelle.

\begin{lstlisting}[language=SQL]
create table mitarbeiter (
  panr int unique not null,
  fachbereich varchar(50) not null,
  monatsgehalt int not null,
  raum varchar(50)
);
\end{lstlisting}

\subsection{Aufgabe 2}

Fügen Sie die Tabelle \lstinline{boat} zur Ihrer Datenbank hinzu. Sie finden die passenden \ac{SQL}-Befehle in einer Textdatei in Microsoft Teams. Sie sehen hier die Struktur der Tabelle.

\begin{lstlisting}[language=SQL]
create table boat (
  boat_id varchar(100) unique not null,
  boat_name text not null,
  color text not null
);
\end{lstlisting}

\subsection{Aufgabe 3}

Für die folgenden Aufgaben sind die Tabellen \lstinline{sailor}, \lstinline{mitarbeiter} und \lstinline{boat} notwendig. Je nach Frage müssen Sie eine andere Tabelle verwenden. Es kann auch sein, dass bei einer Abfrage ein \lstinline[language=SQL]{where} benötigt wird.\\

Erstellen Sie eine \ac{SQL}-Abfrage, welche ...

\begin{enumerate}
\item ...das maximale Alter aller Seeleute ermittelt.
\item ...die Anzahl der roten Boote ermittelt. Die Spalte soll \lstinline{AnzahlRoteBoote} lauten.
\item ...das gesamte Monatsgehalt aller Mitarbeiter berechnet. Die Spalte soll \lstinline{Gesamtgehalt} heissen.
\item ...die Anzahl der verschiedenen Fachbereiche anzeigt.
\item ...das durchschnittliche Monatsgehalt aller Mitarbeiter aus dem Fachbereich Informatik anzeigt. Nennen Sie die Spalte \lstinline{Durchschnittsgehalt}.
\item ...anzeigt, wie viele Seeleute keine Bewertung besitzen.

\end{enumerate}

\newpage

\section{Gruppieren und Gruppen filtern}

Wir können eine \ac{SQL}-Abfrage auch mit Komponenten zur Gruppierung erweitern. Dazu gibt es die \lstinline[language=SQL]{group by} und \lstinline[language=SQL]{having} Komponenten. Allgemein lautet eine \ac{SQL}-Abfrage somit wie folgt:

\begin{lstlisting}[language=SQL]
select spalte(n)
from tabellenname
where bedingung
group by spalte(n)
having bedingung
\end{lstlisting}

Die letzten drei Komponenten sind optional. In der \lstinline[language=SQL]{group by}-Komponente können wir Spalten zur Gruppierung angeben. Dann werden die Zeilen in Gruppen aufgeteilt. In der \lstinline[language=SQL]{select}-Komponente sollten wir dann eine Aggregatfunktion definieren. Diese wird dann auf jede Gruppe angewendet. Möchten wir Gruppen filtern, dann können wir mit der \lstinline[language=SQL]{having}-Komponente eine Bedingung angeben. Nur wenn die Bedingung erfüllt ist, wird eine Gruppe in das Ergebnis aufgenommen. Wie immer können wir auch eine \lstinline[language=SQL]{where}-Komponente verwenden. Diese wird \textbf{vor} der Gruppenbildung aktiv. Zeilen, welche der Bedingung nicht genügen, werden also vor der Gruppenbildung eliminiert.

\subsection{\lstinline[language=SQL]{group by}-Komponente}

Wir zeigen ein paar Beispiele, um das Verhalten der Komponente besser zu verstehen.

\begin{example}[Die Anzahl der Seeleute pro Altersstufe ausgeben]
Es wird nach \lstinline[language=SQL]{age} gruppiert. Für jede Gruppe wird gezählt, wie viele Seeleute in dieser Gruppe vorhanden sind. Neben der Anzahl wird auch das Alter ausgegeben.
\begin{lstlisting}[language=SQL]
select age, count(age) as anzahl
from sailor
group by age
\end{lstlisting}
\end{example}

Wir können natürlich auch andere Aggregatfunktionen und mehrere Aggregatfunktionen pro Gruppe verwenden und nach anderen Spalten gruppieren.

\begin{example}[Die jüngste Person pro Bewertungsstufe ermitteln]
Es wird nach \lstinline[language=SQL]{rating} gruppiert. Für jede Gruppe wird das minimale Alter bestimmt. Ausserdem wird gezählt wie viele Seeleute in einer Gruppe sind. Neben dem Alter wird auch noch das Rating ausgegeben.
\begin{lstlisting}[language=SQL]
select rating, min(age), count(sailor_id)
from sailor
group by rating
\end{lstlisting}
\end{example}

Wir müssen darauf achten, dass in der \lstinline[language=SQL]{select}-Komponente nur diejenigen Spalten aufgelistet werden, welche entweder in der \lstinline[language=SQL]{group by}-Komponente erscheinen oder innerhalb einer Aggregatfunktion benutzt werden. Folgende Abfrage ist \textbf{nicht sinnvoll}:

\begin{example}[Nicht sinnvoll...]
Der \lstinline[language=SQL]{sailor_name} erscheint nicht in der \lstinline[language=SQL]{group by}-Komponente und nicht in einer Aggregatfunktion. Diese Abfrage kann ausgeführt werden, jedoch ist der Wert bzgl. \lstinline[language=SQL]{sailor_name} nicht sinnvoll. MariaDB wählt einen beliebigen Wert.
\begin{lstlisting}[language=SQL]
select sailor_name, rating, min(age), count(sailor_id)
from sailor
group by rating
\end{lstlisting}
\end{example}

\subsection{\lstinline[language=SQL]{having}-Komponente}

Möchten wir nun auch Gruppen filtern, dann können wir die \lstinline[language=SQL]{having}-Komponente verwenden. Hier werden Gruppen nach einer Bedingung gefiltert.

\begin{example}[Die jüngste Person pro Bewertungsstufe ermitteln, falls es mindestens zwei Seeleute pro Bewertungsstufe hat]
Es wird nach \lstinline[language=SQL]{rating} gruppiert. Für jede Gruppe wird gezählt, wie viele \lstinline[language=SQL]{sailor_id}-Werte es gibt. Falls die Anzahl grösser oder gleich 2 ist, dann wird die Gruppe im Ergebnis angezeigt, sonst wird die Gruppe eliminiert (\lstinline[language=SQL]{having}-Komponente). Für die verbleibenden Gruppen wird dann die \lstinline[language=SQL]{select}-Komponente verarbeitet.
\begin{lstlisting}[language=SQL]
select rating, min(age), count(sailor_id)
from sailor
group by rating
having count(sailor_id) >= 2
\end{lstlisting}
\end{example}

Es gilt wieder zu beachten, dass nur diejenigen Spalten in der \lstinline[language=SQL]{having}-Komponente vorkommen dürfen, welche auch in der \lstinline[language=SQL]{group by}-Komponente vorkommen oder Teil einer Aggregatfunktion sind. Wir können natürlich auch alle Komponenten kombinieren, wie folgendes Beispiel zeigt.

\begin{example}[Die jüngste Person pro Bewertungsstufe ermitteln, falls es mindestens zwei Seeleute pro Bewertungsstufe hat. Es sollen nur Personen betrachtet werden, welche bereits 18 Jahre alt sind]
Hier gilt, dass die \lstinline[language=SQL]{where}-Komponente vor der \lstinline[language=SQL]{group by}-Komponente verarbeitet wird. Dies bedeutet, dass zuerst die Seeleute, welche jünger als 18 sind aus dem Ergebnis eliminiert werden.
\begin{lstlisting}[language=SQL]
select rating, min(age), count(sailor_id)
from sailor
where age >= 18
group by rating
having count(sailor_id) >= 2
\end{lstlisting}
\end{example}

\section{Aufgaben}

Für die folgenden Aufgaben sind die Tabellen \lstinline{sailor}, \lstinline{mitarbeiter}, \lstinline{boat} und \lstinline{profile} notwendig. Je nach Frage müssen Sie eine andere Tabelle verwenden. Es kann auch sein, dass bei einer Abfrage ein \lstinline[language=SQL]{where} benötigt wird.\\

Lösen Sie die Aufgaben. Erstellen Sie eine \ac{SQL}-Abfrage, welche ...

\begin{enumerate}
\item ...die Anzahl der Boote pro Farbe ermittelt. Geben Sie die Farbe und die Anzahl aus (Spalte umbenennen).

\item ...pro Beziehungsstatus die Anzahl der Personen ermittelt. Geben Sie auch für diejenigen, die keine Angaben zum Beziehungsstatus machen, die Anzahl aus.

\item ...pro Arbeitgeber und Staat die Anzahl der Personen ermittelt. Geben Sie den Arbeitgeber, den Staat und die Anzahl der Personen aus.

\item ...welche das Durchschnittsgehalt pro Fachbereich ausgibt, jedoch nur dann wenn es über 1000 liegt. Geben Sie das Durchschnittsgehalt und den Fachbereich aus.

\item ...alle Fachbereiche ausgibt, die ein $a$ enthalten und deren Gesamtgehalt (Summe der Monatsgehälter) zwischen 200 und 12000 liegt.

\item ...diejenigen Vornamen ermittelt, welche mehrmals vorkommen. Berücksichtigen Sie nur Vornamen mit einem $C$. Geben Sie den Vornamen und die Anzahl aus.

\end{enumerate}

